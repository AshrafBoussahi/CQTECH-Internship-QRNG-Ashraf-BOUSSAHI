We revised some core Random Number Generation and True Random Number Generation principles to better understand the subject and have a hands-on approach to it, and then we discussed the three cornerstones that make a random number a truly random one. We have also discussed the actual pseudo RNGs and their reproducibility, as well as we gave an example in the appendices of how this problem can be used in the frame of the \textbf{seeds problem}. After that, we have seen how many researchers aimed to overcome these shortcomings by using physical systems (like \textbf{quantum ones}: those that interest us) that have noise as an entropy source and how these methods generate random numbers. Yet, they needed post-processing techniques to extract the uniform randomness we are looking for, and we saw how to evaluate the randomness using \textbf{autocorrelation} and \textbf{Shannon entropy}, as well as evaluating the \textbf{Extraction Efficiency} of the post-processors.

We discussed traditional post-processing techniques and their variants, such as the well-known \textbf{Von Neumann} method, the \textbf{Markov Chain de-correlator}, using the \textbf{Markov model}. We used it to generate some of our test data.

Then we moved on to discovering the main purposes of this study, which is to introduce the novel post-processing technique we named it \textbf{CQTPP} (\textbf{Constantine Quantum Technology Post Processor}) and some of the methods we tried to improve its throughput and compared it with the classical post-processors in order to try to overcome some of their defects, like the \textbf{Von Neumann method} that doesn't perform on inputs with dependencies.

We discussed the \textbf{boson sampling model} and how it is good as an \textbf{entropy source}.

We achieved that the \textbf{CQTPP} succeeded in matters of \textbf{badness} along with the \textbf{Von Neumann post-processor}. In the case of independent raw input, it is preferable to use the \textbf{VNPP} or one of its variants introduced in \cite{zonga}, since their rate can achieve \textbf{62.5\%} when using the \textbf{VN\_8 variant} and is much better than the classical \textbf{CQTPP} (around \textbf{17\%}) and its \textbf{Iter CQTPP}. However, if the input has correlations and has some dependency (\textbf{entanglement between bits}), it is much better to use the \textbf{CQTPP} with a \textbf{Markov chain de-correlator}, where results were near \textbf{0 autocorrelation}, unlike the \textbf{Von Neumann} method, which didn't change anything in terms of autocorrelation. The \textbf{Von Neumann + MKV} performs better but not as well as the \textbf{CQTPP + MKV}.

We also introduced the \textbf{CQT\_RNG2.0 package} and the improvements it contains and can be used and extended for study purposes.

This study aims to inspire and help individuals that don't have previous knowledge about the topic to get introduced to it, and as a trial to achieve better \textbf{decorrelation results} using the \textbf{CQTPost Processor}.
